\documentclass[10pt]{article}
\usepackage[margin=1in]{geometry}
\usepackage{listings}
\usepackage{courier}
\lstset{basicstyle=\footnotesize\ttfamily,breaklines=true}
\begin{document}
\title{ECSE-4965 Term Project}
\author{Dan Foster}
\maketitle
\begin{description}
\item[Title] \hfill \\ %Title
cuxz
\item[Team members] \hfill \\ %Team members - name, email
Dan Foster - \textit{fosted3@rpi.edu}
\item[Proposal] \hfill \\ %What you propose to do: implement (what, on what computer?), survey, ...
Implement LZMA2 (Lempel-Ziv-Markov chain algorithm) using both multiple CPU threads (OpenMP/pthread) and Thrust/CUDA, in the xz file container.
\item[Motivation] \hfill \\ %Why this is interesting. 
Data compression is widely used as a method of saving disk space or reducing the time needed to transfer data between machines. Current compressors like gzip (using DEFLATE) provide fast compression speed, but do not compress data very well; alternative compressors like bzip2 (using the Burrows-Wheeler transform) provide a better data compression at the expense of slower compression speed. xz (using LZMA / LZMA2) provides a very good compression ratio, but is slower than bzip2 and gzip. With the goal of having the best compression ratio, a multi-threaded Thrust/CUDA implementation of the LZMA2 algorithm will provide an excellent compression ratio with an improved compression speed.
\end{description}
\end{document}